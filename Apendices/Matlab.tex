%---------------------------------------------------------------------
%
%                          Ap�ndice B
%
%---------------------------------------------------------------------

\chapter{C�digo MATLAB}
\label{ap2:matlab}

En este ap�ndice se incluye el c�digo MATLAB utilizado en el Cap�tulo 4 para el estudio te�rico de un enlace radioel�ctrico en el mar.

%-------------------------------------------------------------------
\section{freeSpace.m}
%-------------------------------------------------------------------
\label{ap1:freeSpace}
\begin{lstlisting}[language=Matlab, numbers=left]
%---------------------------%
% Ramiro Utrilla Guti�rrez  %
% Julio 2011                %
%---------------------------%
close all
clear all
clc

% RADIO LINK - FREE SPACE PROPAGATION
% Definitions
d = [1000:100:1000*10^3]; % m 
freq = 869.525 * 10^6; % Hz
speed = 3 * 10^8; % m/s
lambda = speed / freq; % m
Pt = 25; % dBm
Gt = 4.5; % dBi
Gr = 4.5; % dBi
Pr = zeros(1,length(d)); % dBm
sens = -104*ones(1,length(d)); % dBm

% Free Space Path Loss
Lbf = zeros(1,length(d));
Lbf = 32.45 + 20*log10(d/(10^3)) + 20*log10(freq/(10^6)); % dB

% Receive Power
Pr = Pt + Gt - Lbf + Gr; % dBm

% Figure 1. Free Space Path Loss
figure
semilogx(d/(10^3), Lbf, 'b', 'linewidth', 2);
grid on
set(gca,'XTickLabel',{'1','10','100','1.000'})
xlabel('\bfDistancia [km]')
ylabel('\bfL_{bf} [dB]')
title(sprintf('P�rdida b�sica de propagaci�n en condiciones de espacio libre \n Frecuencia de operaci�n: 869.525 MHz'));

% Figure 2. Receive Power
figure
semilogx(d/(10^3), Pr, 'b', 'linewidth', 2);
grid on
set(gca,'XTickLabel',{'1','10','100','1.000'})
xlabel('\bfDistancia [km]')
ylabel('\bfP_{r} [dBm]')
title(sprintf('Potencia disponible en el receptor en condiciones de espacio libre \n P_{t}: %d dBm; G_{t}: %g dBi; G_{r}: %g dBi', Pt, Gt, Gr));
hold on
semilogx(d/(10^3), sens, 'r', 'linewidth', 2);
legend('Potencia disponible en el receptor','Sensibilidad del receptor')
hold off
\end{lstlisting}


%-------------------------------------------------------------------
\section{twoRayModel.m}
%-------------------------------------------------------------------
\label{ap1:twoRayModel}
\begin{lstlisting}[language=Matlab, numbers=left]
%---------------------------%
% Ramiro Utrilla Guti�rrez  %
% Julio 2011                %
%---------------------------%
clear all
close all
clc

% RADIO LINK - TWO RAY PROPAGATION MODEL
% Definitions
d = [1:.01:10000]; % m
f = 869.525 * 10^6; % Hz
c = 3 * 10^8; % m/s
lambda = c/f; % m
Pt = 25; % dBm
Gt = 4.5; % dBi
Gr = 4.5; % dBi
ht = 2; % m
hr = 2; % m
beta = pi; % rad
modR = 1;
lb = zeros(1,length(d));
Lb = zeros(1,length(d)); % dB
Lbf = zeros(1,length(d)); % dB
Lb2 = zeros(1,length(d)); % dB
Pr = zeros(1,length(d)); % dBm
sens = -104*ones(1,length(d)); % dBm

% Two Ray Model Path Loss
delta = (4*pi*ht*hr)./(lambda*d);
for i=1:length(d)
    lb(i) = ((4*pi*d(i))/lambda)^2 / (2*(1-cos(delta(i))));
    Lb(i) = 10*log10(lb(i));
end

% Free Space Path Loss
Lbf = 32.45 + 20*log10(d/(10^3)) + 20*log10(f/(10^6)); % dBm

% Two Ray Model Path Loss for long distances
Lb2 = 40*log10(d/(10^3)) - 20*log10(ht*hr) + 120;

% Receive Power
Pr = Pt + Gt - Lb + Gr; % dBm

% Figure 1. Two Ray Path Loss 1
figure
semilogx(d, Lb, 'linewidth', 2)
grid on
set(gca,'XTickLabel',{'1','10','100','1.000','10.000'})
xlabel('\bfDistancia [m]')
ylabel('\bfP�rdida b�sica de propagaci�n [dB]')
title(sprintf('P�rdida b�sica de propagaci�n \n f: 869.525 MHz; h_{t}: %d m; h_{r}: %d m', ht, hr));
hold on
semilogx(d, Lbf,'r', 'linewidth', 2);
legend('Modelo de tierra plana','En condiciones de espacio libre',2)
hold off

% Figure 2. Two Ray Path Loss 2
figure
semilogx(d, Lb, 'linewidth', 2)
grid on
set(gca,'XTickLabel',{'1','10','100','1.000','10.000'})
xlabel('\bfDistancia [m]')
ylabel('\bfP�rdida b�sica de propagaci�n [dB]')
title(sprintf('P�rdida b�sica de propagaci�n \n f: 869.525 MHz; h_{t}: %d m; h_{r}: %d m', ht, hr));
hold on
semilogx(d, Lb2, 'k', 'linewidth', 2);
semilogx(d, Lbf,'r', 'linewidth', 2);
legend('Modelo de tierra plana','Modelo de tierra plana largas distancias','En condiciones de espacio libre',2)
hold off

% Figure 3. Receive Power
figure
semilogx(d/(10^3), Pr, 'b', 'linewidth', 2);
grid on
set(gca,'XTickLabel',{'1','10','100','1.000','10.000'})
xlabel('\bfDistancia [m]')
ylabel('\bfP_{r} [dBm]')
title(sprintf('Potencia disponible en el receptor con modelo de tierra plana \n f: 869.525 MHz; P_{t}: %d dBm; G_{t}: %g dBi; G_{r}: %g dBi; h_{t}: %d m; h_{r}: %d m', Pt, Gt, Gr, ht, hr));
hold on
semilogx(d/(10^3), sens, 'r', 'linewidth', 2);
legend('Potencia disponible en el receptor','Sensibilidad del receptor')
hold off
\end{lstlisting}


%-------------------------------------------------------------------
\section{measurements.m}
%-------------------------------------------------------------------
\label{ap1:measurements}
\begin{lstlisting}[language=Matlab, numbers=left]
%---------------------------%
% Ramiro Utrilla Guti�rrez  %
% Julio 2011                %
%---------------------------%
clear all
close all
clc

% RADIO LINK - TWO RAY PROPAGATION MODEL
% Definitions
d = [1:.001:10]; % m
f = 869.525 * 10^6; % Hz
c = 3 * 10^8; % m/s
lambda = c/f; % m
Pt = 25; % dBm
Gt = 4.5; % dBi
Gr = 4.5; % dBi
ht = 2; % m
hr = 2; % m
beta = pi; % rad
modR = 1;
lb = zeros(1,length(d));
Lb = zeros(1,length(d)); % dB
day1e1 = zeros(1,5); % dBm
day2e1 = zeros(1,5); % dBm
day1e2 = zeros(1,5); % dBm
day2e2 = zeros(1,5); % dBm
day1e3 = zeros(1,5); % dBm
day2e3 = zeros(1,5); % dBm
Pr = zeros(1,length(d)); % dBm
sens = -104*ones(1,length(d)); % dBm
link1 = 1.84*ones(1,5); % km
link2 = 3.06*ones(1,5); % km
link3 = 3.5*ones(1,5); % km

% Two Ray Model Path Loss
delta = (4*pi*ht*hr)./(lambda*(d*10^3));
for i=1:length(d)
    lb(i) = ((4*pi*(d(i)*10^3))/lambda)^2 / (2*(1-cos(delta(i))));
    Lb(i) = 10*log10(lb(i));
end

% Receive Power
Pr = Pt + Gt - Lb + Gr; % dBm

% Measurements
day1e1 = [-95; -93; -96; -103; -97];
day2e1 = [-92; -98; -95; -102; -97];

% Hypothetical Link 2
space1 = Pr(841) - Pr(2061);
day1e2 = day1e1 - space1;
day2e2 = day2e1 - space1;

% Hypothetical Link 2
space2 = Pr(841) - Pr(2501);
day1e3 = day1e1 - space2;
day2e3 = day2e1 - space2;

% Figure 1. Receive Power and Link 1 Measurements
figure
semilogx(d, Pr, 'b', 'linewidth', 2);
grid on
set(gca,'XTickLabel',{'1','10'})
xlabel('\bfDistancia [km]')
ylabel('\bfP_{r} [dBm]')
title(sprintf('Potencia disponible en el receptor. ENLACE 1 \n f: 869.525 MHz; P_{t}: %d dBm; G_{t}: %g dBi; G_{r}: %g dBi; h_{t}: %d m; h_{r}: %d m', Pt, Gt, Gr, ht, hr));
hold on
semilogx(link1, day1e1, 'kx', 'linewidth',2);
semilogx(link1, day2e1, 'mo', 'linewidth',2);
semilogx(d, sens, 'r', 'linewidth', 2);
legend('Modelo de tierra plana','Medidas de la jornada 1','Medidas de la jornada 2','Sensibilidad del receptor')
hold off

% Figure 2. Receive Power and Hypothetical Link 2 Measurements
figure
semilogx(d, Pr, 'b', 'linewidth', 2);
grid on
set(gca,'XTickLabel',{'1','10'})
xlabel('\bfDistancia [km]')
ylabel('\bfP_{r} [dBm]')
title(sprintf('Potencia disponible en el receptor. ENLACE 2 \n f: 869.525 MHz; P_{t}: %d dBm; G_{t}: %g dBi; G_{r}: %g dBi; h_{t}: %d m; h_{r}: %d m', Pt, Gt, Gr, ht, hr));
hold on
semilogx(link2, day1e2, 'kx', 'linewidth',2);
semilogx(link2, day2e2, 'mo', 'linewidth',2);
semilogx(d, sens, 'r', 'linewidth', 2);
legend('Modelo de tierra plana','Medidas hipot�ticas de la jornada 1','Medidas hipot�ticas de la jornada 2','Sensibilidad del receptor')
hold off

% Figure 3. Receive Power and Hypothetical Link 3 Measurements
figure
semilogx(d, Pr, 'b', 'linewidth', 2);
grid on
set(gca,'XTickLabel',{'1','10'})
xlabel('\bfDistancia [km]')
ylabel('\bfP_{r} [dBm]')
title(sprintf('Potencia disponible en el receptor. ENLACE 3 \n f: 869.525 MHz; P_{t}: %d dBm; G_{t}: %g dBi; G_{r}: %g dBi; h_{t}: %d m; h_{r}: %d m', Pt, Gt, Gr, ht, hr));
hold on
semilogx(link3, day1e3, 'kx', 'linewidth',2);
semilogx(link3, day2e3, 'mo', 'linewidth',2);
semilogx(d, sens, 'r', 'linewidth', 2);
legend('Modelo de tierra plana','Medidas hipot�ticas de la jornada 1','Medidas hipot�ticas de la jornada 2','Sensibilidad del receptor')
hold off
\end{lstlisting}
