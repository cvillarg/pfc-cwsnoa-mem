%---------------------------------------------------------------------
%
%                          Cap�tulo 1
%
%---------------------------------------------------------------------

\chapter{Introduction}

\begin{FraseCelebre}
\begin{Frase}
...
\end{Frase}
\begin{Fuente}
...
\end{Fuente}
\end{FraseCelebre}

\begin{resumen}
Esta secci�n tiene que introducir el entorno de radio cognitiva, su
aparici�n, su estado actual, sus caracter�sticas, etc.
Introducir el escenario de CWSN.
Despu�s habr�a que explicar el problema de seguridad en este tipo de
redes --> Motivaci�n
Objetivos
Metodolog�a
Organizaci�n de la tesis
Publicaciones

\end{resumen}


%-------------------------------------------------------------------
\section{Cognitive Radio}
%-------------------------------------------------------------------
\label{sec:cognitive}

Introducci�n general a lo que es Cognitive Radio, la motivaci�n de su
aparici�n, caracter�sticas, estado actual en el mundo, etc.
                                                                                      
%-------------------------------------------------------------------
\section{The scenario: Cognitive Wireless Sensor Networks}
%-------------------------------------------------------------------
\label{sec:cwsn}

Introducci�n espec�fica al escenario que ata�e a esta tesis. Esplicar
las caracter�sticas especiales de este escenario.

%-------------------------------------------------------------------
\section{Motivation: the security in CWSN}
%-------------------------------------------------------------------
\label{sec:securitycwsn}

Introducci�n al problema de seguridad en CWSN. Esplicar
que es un problema m�s importante y cr�tico que en las redes WSN.

%-------------------------------------------------------------------
\section{Objectives}
%-------------------------------------------------------------------
\label{sec:objetives}

Definir los objetivos de la tesis


%-------------------------------------------------------------------
\section{Methodology}
%-------------------------------------------------------------------
\label{sec:methodogy}

Definir la metodolog�a seguida en la tesis


%-------------------------------------------------------------------
\section{Organization}
%-------------------------------------------------------------------
\label{sec:organization}

Organizaci�n del documento


%-------------------------------------------------------------------
\section{Publications}
%-------------------------------------------------------------------
\label{sec:publications}

Las publicaciones que he conseguido en el transcurso de la tesis




%-------------------------------------------------------------------
%\section*{\NotasBibliograficas}
%-------------------------------------------------------------------
%\TocNotasBibliograficas

%Citamos algo para que aparezca en la bibliograf�a\ldots
%\citep{ldesc2e}

%\medskip

%Y tambi�n ponemos el acr�nimo \ac{CVS} para que no cruja.

%Ten en cuenta que si no quieres acr�nimos (o no quieres que te falle la compilaci�n en ``release'' mientras no tengas ninguno) basta con que no definas la constante \verb+\acronimosEnRelease+ (en \texttt{config.tex}).


%-------------------------------------------------------------------
%\section*{\ProximoCapitulo}
%-------------------------------------------------------------------
%\TocProximoCapitulo


% Variable local para emacs, para  que encuentre el fichero maestro de
% compilaci�n y funcionen mejor algunas teclas r�pidas de AucTeX
%%%
%%% Local Variables:
%%% mode: latex
%%% TeX-master: "../Tesis.tex"
%%% End:
