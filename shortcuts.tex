%---------------------------------------------------------------------
%
%                          shortcuts.tex
%
%---------------------------------------------------------------------
%
% Fichero que  declara nuevos comandos LaTeX para evitar la repetición de
% texto y centralizar funcionalidad
%
%---------------------------------------------------------------------
%\def\android{\textit{Android}}
\def\android{Android}
\def\nc{\small{No usado/indiferente}}
\def\messApp{\textit{Messenger} de la aplicaci\'{o}n}
%\def\bt{\textit{Bluetooth}}
\def\bt{Bluetooth}
%\def\wifi{\textit{Wi-Fi}}
\def\wifi{Wi-Fi}
\def\wifiD{Wi-Fi Direct}
\def\adhoc{\textit{Ad-Hoc}}
\def\integer#1{entero#1}
%\def\integers{enteros}
\def\double{\textit{double}}
\def\byteArray{\textit{array} de \textit{bytes}}
%\def\String#1{\textit{cadena#1 de caracteres}}
%\def\Strings{\textit{cadenas de caracteres}}
\def\String#1{cadena#1 de caracteres}
\def\false{\bsq{\textit{false}}}
\def\true{\bsq{\textit{true}}}
\def\bsq#1{\lq{#1}\rq}
%\def\cst#1{{\small\texttt{#1}}}
\def\Lcst#1{\leavevmode\nobreak\hskip0pt plus\linewidth\penalty50\hskip0pt plus-\linewidth\nobreak{\small\texttt{#1}}}% long costant
\def\Scst#1{{\small\texttt{#1}}}% short constant
\def\var#1{\texttt{#1}}
%\def\sphone#1{\textit{smartphone#1}}
%\def\Sphone#1{\textit{Smartphone#1}}
\def\sphone#1{tel\'{e}fono#1 inteligente#1}
\def\Sphone#1{Tel\'{e}fono#1 inteligente#1}
\def\fwork{\textit{framework}}
\def\netbook#1{\textit{netbook#1}}
\def\java{Java}
\def\linux{Linux}

\providecommand{\class}[2][es.upm.die\allowbreak.lsi.pfc\allowbreak.CWSNoA]{%
\texttt{\small\ifthenelse{\equal{#1}{}}{#1\allowbreak.#2}{\ifthenelse{\equal{#1}{util}}{es.upm\allowbreak.die.lsi\allowbreak.pfc\allowbreak.CWSNoA.util\allowbreak.#2}{#1\allowbreak.#2}}%
}%
}
\def\argOne{\textit{arg1}}
\def\argTwo{\textit{arg2}}
\def\obj{\textit{obj}}
\def\replyTo{\textit{replyTo}}
\def\workaround{solucion alternativa}
\def\workaroundCaption{\workaround{} para el}
\def\workaroundCaptionFem{\workaround{} para la}

%---
\def\ListaNodosNoDeliveredItem{
\item[Lista de nodos a los cuales no se han entregado el mensaje] \ListaNodosDescription{}
}
\def\ListaNodosItem{
\item [Lista de nodos] \ListaNodosDescription{}
}
\def\ListaNodosDescription{
Un conjunto de dos \textit{arrays} ligados, uno con los identificadores de los nodos y otro con los nombres de \'{e}stos. De tal forma que el par (identificador, nombre) comparta la misma posici\'{o}n en ambos \textit{arrays}. Se recuperan con las claves ``nodeIdsList'' y ``nodeNamesList'' respectivamente.
}

%---
%-------------------------
% Tabla: estructura mensaje android
%	_init: iniciliza tabla e incluye los campos del mensaje
%	_extra: fila extras incluye un par de valores
%	_extraAddrow: fila adicional
%	_end: finaliza la tabla y e introduce caption y label
%-------------------------
%---------------from---to-----------------------
\def\servToApp{desde el servicio a la aplicaci\'{o}n}
\def\servToApps{desde el servicio a todas las aplicaciones}
\def\appToServ{desde la aplicaci\'{o}n al servicio}
\def\servToBt{desde el servicio al controlador de \bt{}}
\def\btToServ{desde el controlador de \bt{} al servicio}
\def\servToWifi{desde el servicio al controlador de \wifi{}}
\def\wifiToServ{desde el controlador de \wifi{} al servicio}
\def\controllerToServ{desde un controlador (\bt{} o \wifi{}) al servicio}
%---------------from---to-----------------------
\providecommand{\messageTableInit}[7][h]{%
\begin{table}[#1]%
\begin{center}
	\begin{tabular}{|m{3.4cm}|m{2.3cm}|m{2.3cm}|m{2.3cm}|m{2.3cm}|}%
	\hline%
	\multicolumn{5}{c}{\textbf{Campos mensaje #2}} \\%
	\hline%
	\textit{What} & \textit{arg1} & \textit{arg2} & \textit{obj} & \textit{replyTo} \\%
	\hline%
	\parbox[t]{3.4cm}{#3} & #4 & #5 & #6 & #7 \\%
	\hline%
}
\providecommand{\messageTableExtra}[2]{
	\multicolumn{5}{l}{\textit{Extras}}\\%
	\hline%
	\textit{key} & \multicolumn{4}{l|}{\textit{value}}\\%
	\hline
     	{#1} & \multicolumn{4}{p{10.2cm}|}{#2}\\%
	\hline%
}
\providecommand{\messageTableExtraAddrow}[2]{
	#1 & \multicolumn{4}{p{11.2cm}|}{#2}\\%
	\hline%
}
\providecommand{\messageTableEnd}[2]{       
	\end{tabular}%
	\caption{#1} \label{tab:#2}
\end{center} 
\end{table}%
}

\providecommand{\messageTableExtraControllerSpecified}[3]{%
	\noalign{\vskip 0.17cm} %fails with arydshln packet
	\multicolumn{5}{l}{Extras, controlador: {#1}}\\%
	\hline%
	\textit{key} & \multicolumn{4}{l|}{\textit{value}}\\%
	\hline
     	{#2} & \multicolumn{4}{p{10.2cm}|}{#3}\\%
	\hline%
}

\providecommand{\messageTableExtraArbitraryText}[3]{%
	\noalign{\vskip 0.17cm} %fails with arydshln packet
	\multicolumn{5}{l}{#1}\\%
	\hline%
	\textit{key} & \multicolumn{4}{l|}{\textit{value}}\\%
	\hline
     	{#2} & \multicolumn{4}{p{10.2cm}|}{#3}\\%
	\hline%
}
%-------------------------
%-------------------------
% Tabla: estructura mensaje protocol buffer
%	_init: iniciliza tabla e incluye el título del mensaje
%	_end: finaliza la tabla y e introduce caption y label
%-------------------------
\providecommand{\protobufTableInit}[2][h]{%
\begin{table}[#1]%
	\begin{center}%
		\begin{tabular}{c}%
			\textbf{#2, ver figura~\ref{fig:messageProto}}\\ \hline%
		\end{tabular}%
	\end{center}%
%\begin{center}
%Texto encima línea\\
%\noindent\rule{8cm}{0.4pt}
%\line(1,0){450}
%\end{center}
\begin{description}%
}
\providecommand{\protobufItem}[2]{%
\item[\fbox{#1}]\hfill\\[3pt] #2%
}
\providecommand{\protobufTableEnd}[2]{%
\end{description}%
\caption{#1} \label{itm:#2}%
\end{table}%
}
%-----------------------------
% Tabla: 2 columnas API
%      _init: inicializa la tabla
%      _row: fila de la tabla
%      _end: finaliza la tabla con caption y label
%-----------------------------
\providecommand{\apiTableInit}[1]{%
\begin{table}[h]
\begin{center}
	\begin{tabular}{r @{\hskip 0.717cm} l}
	\multicolumn{2}{c}{#1}\\ \hline
	\noalign{\vskip 0.24cm} 
	%\dashuline{Constante usada en el c\'{o}digo} & \dashuline{Representaci\'{o}n en forma de \integer{}} \\  %\cdashline{2-2}
	\uwave{\texttt{Constante usada en el c\'{o}digo}} & \uwave{\texttt{Representaci\'{o}n en forma de \integer{}}} \\  %\cdashline{2-2}
	\noalign{\vskip 0.17cm} %\hdashline
}%
\providecommand{\apiTableRow}[2]{%
	\MakeUppercase{\small #1} &  #2 \\
}%
\providecommand{\apiTableEnd}[2]{%
	\end{tabular}
	\caption{#1} \label{tab:#2}
\end{center}
\end{table}
}%
\definecolor{notImplemented}{gray}{0.7}
\providecommand{\apiTableRowNotImplemented}[2]{%
	\rowstyle{\color{notImplemented} \MakeUppercase{{\small #1}}} & \rowstyle{\color[gray]{0.7}}	 #2 | no implementado\\
}%
%-----------------------------
%	Figura: procesos / mensajes
%				_init: inicializa la figura
%				_initTitle: inicializa la figura e incluye un titulo en la parte superior
%				_end: cierra la figura e incluye caption
%-----------------------------
\providecommand{\umlsdInit}[1][h]{%
\begin{figure}[#1]%
	\begin{center}%
		\begin{sequencediagram}%
}%
\providecommand{\umlsdInitTitle}[2][h]{%
\begin{figure}[#1]%
	\begin{center}%
		\begin{tabular}{c}%
			\textbf{#2}\\ \hline%
			\noalign{\vskip 0.17cm}
		\end{tabular}%
		\begin{sequencediagram}%
}%

\providecommand{\callSelf}[2]{%
\begin{callself}{#1}{\texttt{#2}}{} \end{callself}%
}%

\providecommand{\umlsdEnd}[2]{%
		\end{sequencediagram}
	\end{center}
\caption{#1} \label{fig:#2}
\end{figure}
}%
%-----------------------------
%	Tabla: scheme database
%				_init: inicializa la tabla con el nombre de la tabla en la base de datos
%				_row: fila de la tabla | clave | nombre campo | explicación tipo.
%				_end: cierra la tabla, leyenda e incluye caption
%-----------------------------
\providecommand{\ddbbTableInit}[1]{%
\begin{table}[h]%
\begin{center}%
	\begin{tabular}{c r @{\hskip 7pt} l}%
		\hline%
		\multicolumn{3}{|c|}{Tabla: #1}\\ \hline% 
		\noalign{\vskip 6pt}%
}%

\providecommand{\ddbbTableRow}[3]{%
	\noalign{\vskip 1.17pt}%
	\texttt{#1} \hfill| & #2 & #3 \\\hline%
	\noalign{\vskip 1.17pt}%
}%

\providecommand{\ddbbTableEnd}[2]{%
	\hline%
	\end{tabular}%
	~\\\vspace{7pt}%
	*\texttt{PK}: Primary key, \texttt{FK}: Foreing key\\%
	\caption{#1} \label{tab:#2}%
\end{center}%
\end{table}%
}%
%-----------------------------
% Tabla: 2 columnas state manager
%      _init: inicializa la tabla
%      _row: fila de la tabla
%      _end: finaliza la tabla con caption y label
%-----------------------------
\providecommand{\smTableInit}[1]{%
\begin{table}[h]%
\begin{center}%
	\begin{tabular}{l @{\hskip 1.17cm} l}%
	\multicolumn{2}{c}{#1}\\ \hline
	\noalign{\vskip 0.24cm} 
	%\dashuline{Constante usada en el c\'{o}digo} & \dashuline{Representaci\'{o}n en forma de \integer{}} \\  %\cdashline{2-2}
	\uwave{\texttt{Estados interfaz}} & \uwave{\texttt{Eventos interfaz}} \\  %\cdashline{2-2}
	\noalign{\vskip 0.17cm} %\hdashline
}%
\providecommand{\smTableRow}[2]{%
	\MakeUppercase{\small #1} & \MakeUppercase{\small #2} \\
}%
\providecommand{\smTableEnd}[2]{%
	\end{tabular}%
	\caption{#1} \label{tab:#2}%
\end{center}%
\end{table}%
}%
%-----------------------------
% figura with note: 
%		#1 (optional) where to place
%		#2 image#1 path
%		#3 image#1 size 
%		#4 notes
%		#5 caption
%		#6 label
%-----------------------------
\newcommand{\figuraNota}[6][h!]{%
\begin{figure}[#1]%
\begin{center}%
	\includegraphics[#3]{Imagenes/#2}%
%		\caption{settings1}%

{\footnotesize\sf #4}%
\caption{#5} \label{fig:#6}%
\end{center}%
\end{figure}%
}%
%-----------------------------
% figura side-by-side: 
%		#1 (optional) where to place
%		#2 image#1 path
%		#3 image#1 size 
%		#4 image#2 path
%		#5 image#2 size
%		#6 notes
%		#7 caption
%		#8 label
%-----------------------------
\newcommand{\figuraSideBySide}[8][!h]{%
\begin{figure}[#1]%
\begin{center}%
	\begin{minipage}[t]{.5\textwidth}%
		\centering%
		\includegraphics[#3]{Imagenes/#2}%
%		\caption{settings1}%
	\end{minipage}%
	\begin{minipage}[t]{.5\textwidth}%
		\centering%
		\includegraphics[#5]{Imagenes/#4}%
%		\caption{settings2}%
	\end{minipage}%
	%\caption{settings}%

{\footnotesize\sf #6}%
\caption{#7} \label{fig:#8}%
\end{center}%
\end{figure}%
}%
%-----------------------------
% Tabla: 2 minipages info servicio.
%      _init: inicializa la tabla
%      _row: fila de la tabla
%      _swtich: cambia de minipagina
%      _header: cabecera intermedia
%      _end: finaliza la tabla con caption y label
%-----------------------------
\providecommand{\miniTableInit}[4][h]{%
\begin{table}[#1]%
\begin{center}%
	\begin{tabular}{c}%
	#3\\ \hline%
	\noalign{\vskip 0.24cm}% 
	\begin{minipage}[t]{#2\textwidth}%
		\uwave{\texttt{#4}}%
		\vspace{0.17cm}%
}%
\providecommand{\miniTableRowNormal}[1]{%
		\\ \MakeUppercase{#1}%
}%
\providecommand{\miniTableRow}[1]{%
		\\ \MakeUppercase{\small #1}%
}%
\providecommand{\miniTableRowNI}[1]{%
	\\ \textcolor{notImplemented}{\MakeUppercase{\small #1}} \\[-4pt] {\scriptsize\color{notImplemented}(No implementado)}%
}%
\providecommand{\miniTableSwitch}[3]{%
	\end{minipage}%
	\begin{minipage}[t]{#1\textwidth}%
	\hfill
	\end{minipage}%
	\begin{minipage}[t]{#2\textwidth}%
		\uwave{\texttt{#3}}%
		\vspace{0.17cm}%
}%
\providecommand{\miniTableHeader}[1]{%
		\vspace{0.17cm}\\%
		%\\%
		\uwave{\texttt{#1}}\vspace{0.17cm}%
}%
\providecommand{\miniTableEnd}[2]{%
	\end{minipage}%
	%~\\
	\\
	\noalign{\vskip 0.24cm}% 
	\hline
	%~\\\hline%
	\end{tabular}%
	\caption{#1} \label{tab:#2}%
\end{center}%
\end{table}%
}%
