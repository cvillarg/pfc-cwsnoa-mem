%---------------------------------------------------------------------
%
%                          shortcuts.tex
%
%---------------------------------------------------------------------
%
% Fichero que  declara nuevos comandos LaTeX para evitar la repetición de
% texto y centralizar funcionalidad
%
%---------------------------------------------------------------------

\def\android{\emph{Android}}
\def\nc{\small{No usado/indiferente}}
\def\messApp{Messenger de la aplicaci\'{o}n}
\def\bt{\emph{Bluetooth}}
\def\wifi{\emph{WiFi}}
\def\integer{entero}
\def\double{\emph{double}}
\def\String{\emph{cadena de caracteres}}
\def\false{\emph{false}}
\def\true{\emph{true}}
%-------------------------
% Tabla: estructura mensaje android
%	_init: iniciliza tabla e incluye los campos del mensaje
%	_extra: fila extras incluye un par de valores
%	_extraAddrow: fila adicional
%	_end: finaliza la tabla y e introduce caption y label
%-------------------------
%---------------from---to-----------------------
\def\servToApp{desde el servicio a la aplicaci\'{o}n}
\def\servToApps{desde el servicio a todas las aplicaciones}
\def\appToServ{desde la aplicaci\'{o}n al servicio}
\def\servToBt{desde el servicio al controlador de \bt}
\def\btToServ{desde el controlador de \bt al servicio}
\def\servToWifi{desde el servicio al controlador de \wifi}
\def\wifiToServ{desde el controlador de \wifi al servicio}
\def\controllerToServ{desde un controlador (\bt{} o \wifi{}) al servicio}
%---------------from---to-----------------------
\providecommand{\messageTableInit}[6]{%
\begin{table}[h]%
\begin{center}
	\begin{tabular}{|m{3.4cm}|m{2.3cm}|m{2.3cm}|m{2.3cm}|m{2.3cm}|}%
	\hline%
	\multicolumn{5}{c}{\textbf{Campos mensaje #1}} \\%
	\hline%
	\emph{What} & \emph{arg1} & \emph{arg2} & \emph{obj} & \emph{replyTo} \\%
	\hline%
	\parbox[t]{3.4cm}{#2} & #3 & #4 & #5 & #6 \\%
	\hline%
}
\providecommand{\messageTableExtra}[2]{
	\multicolumn{5}{l}{Extras}\\%
	\hline%
	\emph{key} & \multicolumn{4}{l|}{\emph{value}}\\%
	\hline
     	{#1} & \multicolumn{4}{p{10.2cm}|}{#2}\\%
	\hline%
}
\providecommand{\messageTableExtraAddrow}[2]{
	#1 & \multicolumn{4}{p{11.2cm}|}{#2}\\%
	\hline%
}
\providecommand{\messageTableEnd}[2]{       
	\end{tabular}%
	\caption{#1} \label{tab:#2}
\end{center} 
\end{table}%
}

\providecommand{\messageTableExtraControllerSpecified}[3]{%
	\noalign{\vskip 0.15cm} 
	\multicolumn{5}{l}{Extras, controlador: {#1}}\\%
	\hline%
	\emph{key} & \multicolumn{4}{l|}{\emph{value}}\\%
	\hline
     	{#2} & \multicolumn{4}{p{10.2cm}|}{#3}\\%
	\hline%
}
%-------------------------
%-------------------------
% Tabla: estructura mensaje protocol buffer
%	_init: iniciliza tabla e incluye el título del mensaje
%	_end: finaliza la tabla y e introduce caption y label
%-------------------------
\providecommand{\protobufTableInit}[1]{%
\begin{table}[h]%
	\begin{center}%
		\begin{tabular}{c}%
			\textbf{#1}\\ \hline%
		\end{tabular}%
	\end{center}%
%\begin{center}
%Texto encima línea\\
%\noindent\rule{8cm}{0.4pt}
%\line(1,0){450}
%\end{center}
\begin{description}%
}
\providecommand{\protobufTableEnd}[2]{%
\end{description}%
\caption{#1} \label{itm:#2}%
\end{table}%
}

